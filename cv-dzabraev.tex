%%%%%%%%%%%%%%%%%%%%%%%%%%%%%%%%%%%%%%%%%
% Wilson Resume/CV
% XeLaTeX Template
% Version 1.0 (22/1/2015)
%
% This template has been downloaded from:
% http://www.LaTeXTemplates.com
%
% Original author:
% Howard Wilson (https://github.com/watsonbox/cv_template_2004) with
% extensive modifications by Vel (vel@latextemplates.com)
%
% License:
% CC BY-NC-SA 3.0 (http://creativecommons.org/licenses/by-nc-sa/3.0/)
%
%%%%%%%%%%%%%%%%%%%%%%%%%%%%%%%%%%%%%%%%%

%----------------------------------------------------------------------------------------
%	PACKAGES AND OTHER DOCUMENT CONFIGURATIONS
%----------------------------------------------------------------------------------------

\documentclass[10pt]{article} % Default font size

\input{structure.tex} % Include the file specifying document layout

\usepackage{graphicx}
\graphicspath{ {./} }
%----------------------------------------------------------------------------------------

\begin{document}

%----------------------------------------------------------------------------------------
%	NAME AND CONTACT INFORMATION
%----------------------------------------------------------------------------------------

\title{Maksim Dzabraev -- Résumé} % Print the main header

%------------------------------------------------

\parbox{0.5\textwidth}{ % First block
\begin{tabbing} % Enables tabbing
\hspace{3cm} \= \hspace{4cm} \= \kill % Spacing within the block
{\bf Address} \> B1656left, 1, Leninskie Gory,\\ % Address line 1
\> Moscow, Russia \\ % Address line 2
{\bf Date of Birth} \> 19$^{th}$ February 1993 \\ % Date of birth 
{\bf Languages} \> Russian (Native), English (technical)\\ % Nationality
{\bf Mobile Phone} \> +7 926 733 5139 \\ % Mobile phone
{\bf Email} \> \href{mailto:dzabraew@gmail.com}{dzabraew@gmail.com} \\ % Email address
{\bf Github} \> \href{https://github.com/dzabraev}{github.com/dzabraev} \\ % Mobile phone
\end{tabbing}}
\hfill % Horizontal space between the two blocks
\parbox{0.5\textwidth}{ % Second block
\begin{tabbing} % Enables tabbing
\hspace{3cm} \= \hspace{4cm} \= \kill % Spacing within the block
\
\includegraphics[scale=0.12]{photo}
\end{tabbing}}
%	EDUCATION SECTION
%----------------------------------------------------------------------------------------

\section{Education}

\tabbedblock{
\bf{2017-Present} \>PhD student - \href{http://www.msu.ru/en}{Lomonosov Moscow State University} \\[5pt]
\>\+
\textit{Faculty of Mechanics and Mathematics, Department of Computational Mathematics}
}

\tabbedblock{
\bf{2011-2017} \> Specialist in Mathematics - \href{http://www.msu.ru/en}{Lomonosov Moscow State University} \\[5pt]
\>\+
\textit{Faculty of Mechanics and Mathematics, Department of Computational Mathematics}
}

% \section{Diploma}
%
% There were two problems I investigated in my \href{https://github.com/dzabraev/diploma/blob/master/paper/diploma.pdf}{thesis}.
% The first problem was to extract names of science seminars from texts written in arbitrary language. I proposed an approach to solve it with neural networks.
% The second problem was to make a sentiment classifier for articles from mass media. I used SVM and applied it to estimate MSU reputation.

\section{Listened courses}
\begin{itemize-noindent}
\item{Mazurenko I.L., Mathematical foundations of signal processing}
\item{\href{https://www.coursera.org/specializations/machine-learning-data-analysis}{Introduction to machine learning}}
\end{itemize-noindent}


%----------------------------------------------------------------------------------------
%	EMPLOYMENT HISTORY SECTION
%----------------------------------------------------------------------------------------

\section{Employment History}
\job
{Apr 2013 -}{Present}
{NIISI RAS, Moscow, Russia}
{https://www.niisi.ru/}
{Developer, engineer}
{

%Обязаности
%* Поддержка и реализация новых возможностей в отладчике для ОСРВ.
%* Своевременное портирование отладчика для ОСРВ на самую новую версию GDB.
%* Поддержка и реализация новых возможностей в агенте отладки, который находится в рамках ОСРВ.
%* Общение с разработчиками ОСРВ и пользователями отладчика на предмет 

%Accomplishments
%* К настоящему моменту сделано, что отладчик ОСРВ базируется на версии gdb 8.1.
%* Реализована CI-система тестов отладчика.
%* Реализован вспомогательный tui-интерфейс для отладчика. Просмотр исходного
%кода, просмотр/изменение регистров, локальных переменных, просмотр и переключение
%потоков и фреймов.

\begin{itemize-noindent}
\subsection{Responsibilities}
\item{Maintaining and developing new features for RTOS Debugger (Debugger is based on GNU gdb).}
\item{Keeping RTOS debugger up-to-date with the latest gdb version.}
\item{Maintaining and developing new features for debugger stub (stub is the part of RTOS Baget, handles requests from debugger).}
\item{Building rpm package for each new RTOS Debugger release.}

\subsection{Accomplishments}
\item{RTOS Debugger was ported to gdb 8.1.}
\item{Configured Gitlab CI to run tests on vmips and some real hardware. The solution is based on GNU Dejagnu.}
\item{Implemented \href{https://github.com/dzabraev/mcgdb}{tui} for RTOS Debugger.}
\item{Contributed to the development of vmips gdb stub. Vmips gdb stub allows debug RTOS at CPU level.}
\end{itemize-noindent}

\rule{0mm}{5mm}\textbf{Technologies:} RTOS, MIPS, C, Python, GDB, GDB remote-protocol, CPU Simulation, Vmips}

%------------------------------------------------

\job
{Sep 2016 -}{Present}
{IAS Istina, Moscow, Russia}
{https://istina.msu.ru/}
{Front-end/backend-end developer}
{

% Обязаности.
% * В ИАС ИСТИНА существуют подсистема учета научной деятельности относительно научного оборудования.
%       Я должен заниматься поддержкой данной подсистемой и разработкой нового функционала.
% * В ИАС ИСТИНА существует подсистема учета выступлений научных сотрудников в СМИ.
%       Я должен исправлять ошибки в этой подсистеме.
% * Добавление новых возможностей и исправление ошибок в любой подсистеме, по требованию
%       руководства.
% * Участие в обсуждениях приоритетных путей развития системы.
%
% Достижения.
% * Реализован интерфейс (activity-block) представляющий следующий спектр возможностей:
%       * Прикреплять научные результаты к объектам системы. Например, прикрепление статьи к прибору.
%       * При помощи данного интерфейса ответственный может подтвердить или отклонить прикрепленный результат.
%       * Поиск по прикрепленным резульатам.
%   Сейчас этот интерфейс используется в нескольких подсистемах ИАС ИСТИНА.
% * Был осуществлен перенос модуля с django-select2 с версии 4.x на 5.х. Эти версии имеют разное API,
%       в связи с этим потребовалось переписать большое количество кода.
% * Были реализованы selenium-тесты для activity-block и автоматические тесты для подсистемы оборудования.
% * Принимал участие в разработки CI системы тестирования.
% * Разработан crawler на основе Firefox и реализован алгоритм, который позволяет заходить
%       во всевозможные места на произвольной веб-странице и извлекать необходимые данные.
% * Разработка анализатора публикаций в СМИ на предмет ухудшение/улучшения репутации МГУ.


\begin{itemize-noindent}
\subsection{Responsibilities}
\item{maintaining and improving functionality of scientific activity monitoring system}

\subsection{Achievements}
\item{Developed functional tests of complex user interface with selenium.}
\item{Ported legacy code from 4.x to 5.x version of django-select2.}
\item{Configured CI for the project.}
\item{Introduced new development environment, based on \href{https://nixos.org/nix/}{NIX} package manager.}
\item{Developed universal web crawler for retrieving texts from arbitrary mass media web-sites.}
%\item{Developed crawler based on Firefox (javascript execution) and algorithm allows traverse arbitrary web-page.
%Several millions articles was extracted by crawler from mess media web-sites.}
\item{Implemented sentiment analysis system revealing MSU reputation according to the news feed.}
%\item{Developed sentiment analyzer for classification articles about Lomonosov MSU.
%Analyzer predicts one of three labels +,-,0. Where + means that article increases
%reputation of MSU, - descrease, and zero means not affecting to reputation.}
\end{itemize-noindent}

\rule{0mm}{5mm}\textbf{Technologies:} HTML, CSS, Javascript, Python, Django, NIX package manager/NIXOS, Oracle, PostgreSQL, Selenium}




%----------------------------------------------------------------------------------------
%	IT/COMPUTING SKILLS SECTION
%----------------------------------------------------------------------------------------

\section{Software Engineering Skills}

\skillgroup{Programming Languages}
{
\textit{Python}\\
\textit{C/C++}\\
\textit{Bash}\\
\textit{Nix expression language}
}

%------------------------------------------------

%------------------------------------------------

\skillgroup{Technologies}
{
\textit{Linux}\\
\textit{scikit-learn, pandas, numpy, keras}
\textit{git, redmine, sentry}\\
\textit{gdb, valgrind, gprof}\\
\textit{SQL, Oracle, PostgreSQL, Sqlite}\\
\textit{Django, HTML, CSS, JavaScript/jQuery}
}

\section{Contribution to Open Source}

\subsection{merged}
\href{https://github.com/bminor/binutils-gdb/commit/484d8d361de65a8489252d14511b77c142d859a1}{GDB, Made gdb.selected\_thread().inferior return a new reference}\\
\href{http://midnight-commander.org/ticket/3870?cversion=0&cnum_hist=1}{midnight commander, fixed incorrect variable usage in quick widget}\\
\href{https://github.com/mesonbuild/meson/pull/2976}{Mesonbuild, include\_directories fixed order}\\
\href{https://github.com/NixOS/nixpkgs/pull/33715}{Nixpkgs, did nixification for diff\_cover, pydocstyle, jinja2\_pluralize}\\
\href{https://github.com/orivej/pybfd/commit/f5ed88590c66a3570b7e5b9aa4c723c40686271d}{pybfd, fixed python binding (bfd\_my\_archive(abfd) was removed)}\\
\href{https://github.com/dzabraev/CUDA-grep/commit/79f4fddfab3cc7950cbe6c9b3a54bcb2b87dc3cc}{CUDAGrep, made cudagrep work on cuda5}

\subsection{submitted}

\href{https://sourceware.org/ml/gdb-patches/2018-01/msg00415.html}{GDB, fixed segfault error when running gdb on gcc7 produced debug info}


%----------------------------------------------------------------------------------------

\end{document}